\documentclass{article}

\begin{document}

\title{Energy Density in an Electromagnetic Wave}
\author{Divyaratna Joshi}
\date{\today}

\maketitle
s
\section{CH22B064}
\subsection{Equation for Energy in an Electromagnetic Wave}
\footnote{\bibliography{bblgrphy}}

The total energy density (u) in an electromagnetic wave is given by the following equation:

\[
u = \frac{{\varepsilon_0 E^2}}{2} + \frac{{B^2}}{2\mu_0}
\]

Where,
\\ $\varepsilon_0$: permittivity of free space
\\ E: Electric field strength
\\ B: Magnetic field strength
\\ $\mu_0$: Permeability of free space.

\subsection{What is Energy in an Electromagnetic Wave?}

\\ The Energy Density of an Electromagnetic Wave refers to the amount of energy carried by the wave in a unit volume region of space at a given time. The first term, $\frac{{\varepsilon_0 E^2}}{2}$, represents energy associated with electric fields, while the second term, $\frac{{B^2}}{2\mu_0}$, represents the energy associated with magnetic fields. The synchronization of these two components propagates the wave through space. Understanding Energy Density of electromagnetic waves plays a crucial role in understanding the interaction of waves in optics, wireless communication and transmission. We recently came across this formula and concept in our physics course (PH1020) in Electrodynamics. I chose the equation for the same reason and the fact that it includes lots of new symbols. 
\\ -By Divyaratna Joshi (CH22B064) github id- @akulsylvania



\end{document}
