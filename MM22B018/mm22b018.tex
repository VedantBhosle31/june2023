\documentclass{article}
\usepackage{hyperref}
\title{\vspace{-5cm}The Ehrenfest Theorem}
\author{}
\date{\today}

\begin{document}
\maketitle



\section{MM22B018}
NAME - Apratim Mahapatra \\
GITHUB USER-ID - ApratimMahapatra

\subsection{The Ehrenfest Theorem}
The Ehrenfest theorem reconciles classical and quantum mechanics and thus validates the existence of quantum mechanics while accepting the principles of classical mechanics.\\\\
The statement of the theorem is as follows: \\
"\emph{Quantum mechanics gives the same results as classical mechanics for a particle for which average or expectation values of dynamical quantities are involved.}"\footnote{\url{https://www.physicsvidyapith.com/2023/03/ehrenfests-theorem-and-derivation.html}} \\
\\
The equations for the theorem are: \\ 
$$m \frac{d\langle x\rangle}{dt} = \langle p\rangle$$
$$\frac{d\langle p\rangle}{dt} = - \langle \frac{dV}{dx}\rangle$$
\\\\
Here, 
$$\langle x\rangle = \int ^{\infty} _{- \infty} x |\psi|^2 dx$$
$$\langle p\rangle = - i \hbar \int ^{\infty} _{- \infty} {\psi}^* \frac{\partial \psi}{\partial x} dx$$
\\\\
Seeing as the theorem implies that $\langle x\rangle$ plays the role of particle displacement, this means that if the extent of the wave function $\psi$ is negligible then a measurement of x gives a value that is close to $\langle x \rangle$. Thus, quantum mechanics corresponds to classical mechanics in the limit that the spatial extent of the wave function is negligible. \footnote{\url{https://phys.libretexts.org/Bookshelves/Quantum_Mechanics/Introductory_Quantum_Mechanics_(Fitzpatrick)/03\%3A_Fundamentals_of_Quantum_Mechanics/3.04\%3A_Ehrenfest's_Theorem}}
\subsection{Reason for this theorem:}
As strange as it is, I had first heard of this theorem in a television comedy show. It sounded fascinating to me and upon looking it up, I realised its significance as a theorem to connect quantum and classical mechanics. 

\end{document}