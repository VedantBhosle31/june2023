\documentclass[12pt]{article}
\usepackage{amsmath}
\usepackage{graphicx}

\author{CH22B070}
\title{Equation}

\begin{document}
\maketitle

\section{CH22B070}

Name: Joshita Deka \\
GitHub Username: joshitad


\subsection{Introduction}

Here, I will introduce another equation called Euler's formula, which is a fundamental result in complex analysis.

\subsection{The Equation}

Euler's formula can be written in different forms. One common form is:

\begin{align*}
    e^{i\theta} &= \cos(\theta) + i\sin(\theta)
\end{align*}

Alternatively, we can express Euler's formula using the exponential function:

\begin{align*}
    e^{i\theta} &= \exp(i\theta)
\end{align*}

where:
\begin{itemize}
    \item \(e\) is the base of the natural logarithm,
    \item \(i\) is the imaginary unit (\(i^2 = -1\)),
    \item \(\theta\) is the angle in radians.
\end{itemize}
\begin{figure}[htp]
    \centering
    \includegraphics[width=4cm]{image}
    \caption{Vector DEscription of the Eulers formula}
    \label{fig: Eulers vector formula}
\end{figure}

This equation shows the profound connection between the exponential function, trigonometric functions, and complex numbers.\cite{ref=inbook}

\subsection{Applications}

Euler's formula has numerous applications in mathematics, physics, and engineering. It plays a vital role in complex analysis, Fourier analysis, signal processing, and quantum mechanics. The formula provides a powerful tool for simplifying calculations involving complex numbers and understanding the behavior of periodic functions.
\bibliography{citation-1}
\bibliographystyle{plain}

\end{document}
