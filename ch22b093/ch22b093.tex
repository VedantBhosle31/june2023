\documentclass{article}
\usepackage{graphicx} % Required for inserting images
\usepackage{amsmath}
\title{BINOMIAL EQUATION}
\author{Roopesh P ch22b093}
\date{June 2023}

\begin{document}

\maketitle

\section{BINOMIAL THEOREM}
Binomial theorem primarily helps to find the expanded value of the algebraic expression of the form $(x + y)^n$. Finding the value of $(x + y)^2$, $(x + y)^3$, $(a + b + c)^2$ is easy and can be obtained by algebraically multiplying the number of times based on the exponent value. But finding the expanded form of (x + y)17 or other such expressions with higher exponential values involves too much calculation. It can be made easier with the help of the binomial theorem.

The exponent value of this binomial theorem expansion can be a negative number or a fraction. Here we limit our explanations to only non-negative values. Let us learn more about the terms, formula and the properties of coefficients in this binomial expansion article.
\section{HISTORY}
The binomial theorem for positive integral exponents was discovered in Europe in the sixteenth century. The triangular array formed by the binomial coefficients undoubtedly played a very important role in the development. The array was known as Pascal triangle (+ A.D. 1665) in Europe. It appeared originally in the work of Apianus (+1527), Stifel (+1544), Scheubel (+1545), Tartaglia (+1556), Bombelli (+1572) and others. The same array was known in China as the 'Old method chart of seven multiplying squares' and appeared at least two centuries earlier in the work of Chu Shih-Chieh (+1303), Yang Hui (+1261) and Chia Hsien (+1100). The paper, apart from early discovery of the theorem in India, shows that the same triangular array was known as meru-prastara in India and occurs several centuries earlier than that of China.\cite{bag1966binomial}
\section{WHAT IS BINOMIAL THEOREM?}
The binomial theorem states the principle for expanding the algebraic expression $(x + y)^n$ and expresses it as a sum of the terms involving individual exponents of variables x and y. Each term in a binomial expansion is associated with a numeric value which is called coefficient.
The binomial coefficient, \(\binom{n}{k}\), is defined by the expression:
\[
    \binom{n}{k} = \frac{n!}{k!(n-k)!}
\]
Statement: According to the binomial theorem, it is possible to expand any non-negative power of binomial $(x + y)^n$ into a sum of the form,

$(x+y)^n$ = \(\binom{n}{0}\) $x^n$+ \(\binom{n}{1}\) $x^(^n^-^1^)y$ + \(\binom{n}{2}\) $x^(^n^-^2^) y^2$ + ... + \(\binom{n}{n-1}\) $xy^(^n^-^1^)$ + \(\binom{n}{n}\) $y^n$

where,$ n\geq 0$ is an integer and each \(\binom{n}{k}\) is a positive integer known as a binomial coefficient.

Note: When an exponent is zero, the corresponding power expression is 1. This multiplicative factor is often omitted from the term, therefore often the right hand side is directly written as \(\binom{n}{0}\) $x^n$ + .... . This formula is also referred to as the binomial formula or the binomial theorem.
\\
Example: Let us expand $(x+3)^5$ using the binomial theorem. Here y = 3 and n = 5. Substituting and expanding, we get:

$(x+3)^5$ =  \(\binom{n}{0}\) $x^5 3^0$ +  \(\binom{n}{1}\) $x^(^5^-^1^) 3^1$ +  \(\binom{n}{2}\) $x^(^5^-^2^) 3^2$ +  \(\binom{n}{3}\) $x^(^5^-^3^) 3^3$ +  \(\binom{n}{4}\) $x^(^5^-^4^) 3^4$ +  \(\binom{n}{5}\) $x^(^5^-^5^) 3^5$

= $x^5$ + 5$x^4$(3) + 10$x^3$(9) + 10 $x^2$(27) + 5x (81) + 243
\\
= $x^5$ + 15 $x^4$ + 90 $x^3$ + 270 $x^2$ + 405 x + 243
\section{BINOMIAL THEOREM PROOF}
Let x, a, $n \in N$. Let us prove the binomial theorem formula through the principle of mathematical induction. It is enough to prove for n = 1, n = 2, for n = $k \geq 2$, and for n = k+1.

It is obvious that $(x+y)^1$ = x+y and

$(x+y)^2$ = (x+y)(x+y)

= $x^2$ + xy + xy + $y^2$ (using distributive property)

= $x^2$ + 2xy + $y^2$

Thus the result is true for n = 1 and n = 2. Let k be a positive integer. Let us prove the result is true for $k \geq 2$.

Assuming $(x + y)^n$ = \sum  \(\binom{n}{r}\)  $x^(^n^-^r^) y^r$,

$(x + y)^k$ = \sum \(\binom{k}{r}\)  $x^(^k^-^r^) y^r$

⇒ $(x+y)^k$ =  \(\binom{k}{0}\)  $x^k y^0$ +  \(\binom{k}{1}\)  $x^(^k^-^1^) y^1$ +  \(\binom{k}{2}\)  $x^(^k^-^2^) y^2$ + ... +  \(\binom{k}{r}\)  $x^(^k^-^r^) y^r$ +....+  \(\binom{k}{k}\)  $x^0y^k$

⇒ $(x+y)^k$ = $x^k$ + \(\binom{k}{1}\) $x^(^k^-^1^) y^1$ + \(\binom{k}{2}\) $x^(^k^-^2^) y^2$ + ... + \(\binom{k}{r}\) $x^(^k^-^r^) y^r$ +....+ $y^k$

Thus the result is true for n = $k \geq 2$.

Now consider the expansion for n = k + 1.

$(x + y)^(^k^+^1^)$ = (x + y) $(x + y)^k$

= (x + y) ($x^k$ + \(\binom{k}{1}\) $x^(^k^-^1^) y^1$ + \(\binom{k}{2}\) $x^(^k^-^2^) y^2$ + ... + \(\binom{k}{r}\) $x^(^k^-^r^) y^r$ +....+ $y^k$)

= $x^(^k^+^1^)$ + (1 + \(\binom{k}{1}\))$x^k y$ + (\(\binom{k}{1}\) + \(\binom{k}{2}\)) $x^(^k^-^1^) y^2 + ... + (\(\binom{k}{r-1}\) + \(\binom{k}{r}\)) $x^(^k^-^r^+^1^) y^r$ + ... + (\(\binom{k}{k-1}\) + 1) $x y^k$ + $y^k$+1

= $x^(^k^+^1^)$ + \(\binom{k+1}{1}\)$x^k y$ + \(\binom{k+1}{2}\) $x^(^k^-^1^) y^2$ + ... + \(\binom{k+1}{r}\) $x^(^k^-^r^+^1^) y^r$ + ... + \(\binom{k+1}{k}\) $x y^k$ + $y^k$+1 [Because \(\binom{n}{r}\) + \(\binom{n}{r-1}\) = \(\binom{n+1}{r}\)]

Thus the result is true for n = k+1. By mathematical induction, this result is true for all positive integers 'n'. Hence proved.
\section{PROPERTIES OF BINOMIAL THEOREM}
\begin{itemize}
    \item The number of coefficients in the binomial expansion of $(x + y)^n$ is equal to (n + 1).
    \item There are (n+1) terms in the expansion of $(x+y)^n$.
    \item The first and the last terms are $x^n$ and $y^n$. respectively.
    \item From the beginning of the expansion of $(x + a)^n$, the powers of x, decrease from n up to 0, and the powers of a, increase from 0 up to n.
    \item In the binomial expansion of $(x + y)^n$, the rth term from the end is (n – r + 2)th term from the beginning.
    \item If n is even, then in $(x + y)^n$ the middle term = (n/2)+1 and if n is odd, then in (x + y)n, the middle terms are (n+1)/2 and (n+3)/2.
    
\end{itemize}
\section{BINOMIAL EQUATION FOR NEGATIVE POWERS}
Newton (1676) showed the formula also holds for negative integers -n,

 $(x+a)^(^-^n^)$=\sum  \(\binom{-n}{k}\) $x^k$ $a^(^-^n^-^k^)$,
 \\
which is the so-called negative binomial series and converges for $|x|$ \leq a.\cite{weisstein2002binomial}
\bibliographystyle{plain}
\bibliography{ref,name}
\end{document}
