\documentclass{article}
\usepackage{hyperref}
\usepackage{amsmath}

\begin{document}

\section{AE22B012}
\textbf{Name:} [Siddhesh dileep yadav]

\textbf{GitHub User-id:} [siddhesh2k4]

\subsection{Steady Flow Energy Equation}

The steady flow energy equation states that for a steady mass flow, the subtraction of the rate change of heat transfer and work input (except flow work) equals the subtraction of the summation of the rate change of mass times outlet enthalpy, kinetic energy, potential energy, etc., and the summation of the rate change of mass times outlet enthalpy, kinetic energy, and potential energy. Mathematically, this equation can be represented as:

\begin{equation}
    \frac{d(Q - W')}{dt} = \frac{dm}{dt} \left( (h_e + \frac{{v_e^2}}{2} + gze) - (h_i + \frac{{v_i^2}}{2} + gzi) \right)
\end{equation}

where:
\begin{align*}
    & d(Q - W')/dt \quad \text{is the rate of change of heat transfer and work input} \\
    & dm/dt \quad \text{is the rate of change of mass flow} \\
    & h_e, v_e, gze \quad \text{are the outlet enthalpy, velocity, and elevation} \\
    & h_i, v_i, gzi \quad \text{are the inlet enthalpy, velocity, and elevation}
\end{align*}

This equation describes the conservation of energy for a steady flow system, taking into account the changes in heat transfer, work, and different forms of energy along the flow path.

\footnote{\url{https://en.wikipedia.org/wiki/Flow_process}}
\end{document}
