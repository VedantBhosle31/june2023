\documentclass{article}
\usepackage{graphicx} % Required for inserting images

\title{GENERALISED EQUATION FOR SHM}
\author{U.Ashwin }
\date{June 2023}

\begin{document}

\maketitle
\begin{center}
    user id ; ashwinch22b025
\end{center}
\section{Introduction}\cite{morin2008introduction}
A Simple Harmonic Motion, or SHM, is defined as a motion in which the restoring                    force is directly proportional to the displacement of the body from its mean position. The direction of this restoring force is always towards the mean position.It is a special case of oscillation, along with a straight line between the two extreme points (the path of SHM is a constraint)
\section{Types of shm}
The SHM, or Simple Harmonic Motion, can be classified into two types:
\begin{itemize}
    \item Linear SHM
    \item Angular SHM
\end{itemize}
\subsection{Linear SHM}
When a particle moves to and fro about a fixed point (called equilibrium position) along with a straight line, then its motion is called linear Simple Harmonic Motion.

For example, the spring-mass system

\textbf{Conditions for Linear SHM}
The restoring force or acceleration acting on the particle should always be proportional to the displacement of the particle and directed towards the equilibrium position.
\begin{figure}[h!]
    \centering
    \includegraphics[width=\linewidth]{Screenshot (56).png}
    \caption{Linear shm}
    \label{fig:enter-label}
\end{figure}

\subsection{Angular SHM}
When a system oscillates angular long with respect to a fixed axis, then its motion is called angular simple harmonic motion.

\textbf{Conditions to Execute Angular SHM}
The restoring torque (or) angular acceleration acting on the particle should always be proportional to the angular displacement of the particle and directed towards the equilibrium position.
\begin{figure}[h]
    \centering
    \includegraphics[width=\linewidth]{Screenshot (57).png}
    \caption{Angular shm}
    \label{fig:enter-label}
\end{figure}
\section{Simple Harmonic Motion Equation and Its Solution}
\subsection{Linear SHM}
\begin{equation}
d^2x/dt^2 + \omega^2x = 0
\end{equation}
The solution for the above equation is
\begin{equation}
    x=Asin((\omega)t+\phi)
\end{equation}
   
    $\omega$ is the angular frequency

    A is the amplitude
    
    t is the time taken

    
    $\phi$ is the initial phase
\subsection{Angular SHM}
\begin{equation}
    d^2\theta/dt^2 + \omega^2\theta = 0
\end{equation}
The solution for the above equation is
\begin{equation}
     \theta=(\lambda)sin((\omega)t+\phi)
\end{equation}

    $\omega$ is the angular frequency

    $\lambda$ is the amplitude
    
    t is the time taken

    
    $\phi$ is the initial phase
\subsection{Damped harmonic motion}
\begin{equation}
    d^2x/dt^2 + \gamma dx/dt +\omega^2x=0

    
    \gamma=p/m=force/velocity*mass

    
    \omega^2=k/m
\end{equation}
The solution for the above equation is 
\begin{figure}[h]
    \centering
    \includegraphics[width=\linewidth]{Screenshot (68).png}
    \caption{Damped Harmonic Oscillator}
    \label{fig:enter-label}
\end{figure}
\bibliographystyle{plain}
\bibliography{References}
\end{document}


