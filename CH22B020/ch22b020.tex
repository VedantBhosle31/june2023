\documentclass[12pt,a4paper]{article}
\usepackage{graphicx} 
\usepackage{amsmath}
\usepackage{amsfonts}
\usepackage{amssymb}
\usepackage{url}
\usepackage[left=2cm,right=2cm,top=3.5cm,bottom=5cm]{geometry}

%------------------------
\title{ID2090 : Assignment 4}
\author{Deepanjhan Das : CH22B020}
\date{June 18, 2023}
%------------------------

\begin{document}
%------------------------
\maketitle
%------------------------
\thispagestyle{empty}
\vspace{1cm}
\section{CH22B020}
\subsection{Student-Info}
\textbf{FULL NAME} : Deepanjhan Das \\
\textbf{GITHUB ID} : deep183Das \\
\textbf{GITHUB PAGE} : \url{https://github.com/deep183Das}  \\
\textbf{EQUATION CHOOSEN} : Black-Scholes Equation (Finance) \\

\subsection{Introduction}

The \textbf{Black-Scholes model} is a mathematical model used to calculate the theoretical price of financial options. The model provides insights into option pricing and has become a pivotal point of modern finance.

\subsection{Discussion}

In the year of 1973, Fischer Black and Myron Scholes formulated a mathematical formula, known as \textbf{Black-Scholes equation(BSe)}, to calculate the price of a European-style call or put option. In an article by Saima Rashid~\footnotemark[1], he refers about this equation by saying that ``The pioneering Black-Scholes equation (BSe) is at the core
of contemporary financial economics, and it is indeed tough
to communicate about mainstream capitalism without mentioning the revolutionary BSe."


The \textbf{Black-Scholes equation}(BSe) is a partial differential equation which is as follows~\footnotemark[2] :
%-------------
\begin{equation}
    \textbf{D}_\rho^\delta\upsilon + \frac{{1}}{{2}}\omega^2S^2\frac{{\partial^2\upsilon}}{{\partial S^2}} + \zeta S\frac{\partial\upsilon}{{\partial S}} - \zeta\upsilon = 0
    \label{eq1}
\end{equation}

The fractional interpretation of BSe is characterized in financial services by eq~\ref{eq1}, which is subject to the playoff mapping~\footnotemark[2] :

\begin{equation}
    \upsilon(S,\xi) = \text{ max} (S - E,0)
    \label{eq2}
\end{equation}
%-----------
where:

\begin{itemize}
\item $\upsilon$ denotes the alternative means worth at underlying asset price(S) of the moment($\rho$).
\item $\xi$ indicates the termination term.
\item E symbolizes to share value.
\item $\textbf{D}_\rho^\delta\upsilon$ denotes the partial derivative of $\upsilon$ with respect to moment $\rho$, capturing the time decay or change in the option price over time.
\item $\frac{{\partial^2\upsilon}}{\partial S^2}$ represents the second partial derivative of $\upsilon$ with respect to the underlying asset price(S), measuring the curvature of option price.
\item The parameter $\zeta$ represents the uncertainty of borrowing until it matures.
\item The continual $\omega$ indicates the unpredictability of a trading asset.
\end{itemize}


It is to be mentioned that $\upsilon(0,\rho)$ = 0 and $\upsilon(S,\xi)$ $\approx$ S as S $\rightarrow$ $\infty$. 

\subsection{Assumptions}

The Black-Scholes equations takes certain assumptions which includes the efficient market hypothesis, no transaction costs or taxes, constant volatility and the ability of borrowing and lending at the risk-free interest rate. 

\subsection{Conclusion}

Though this model has its limitations, it has had a great impact on the field of financial mathematics and option pricing. 

%----------------------
\footnotetext[1]{Saima Rashid, "An Efficient Method for Solving Fractional Black-Scholes Model with Index and Exponential Decay Kernels" 2022, Journal of Function Spaces}

\footnotetext[2]{Necati {\"O}zdemir and Mehmet Yavuz, "Numerical Solution of Fractional Black-Scholes Equation by Using the Multivariate Pad{\'e} Approximation", Volume-132, Pages-1050-1053, Journal of Acta Physica Polonica A}

\end{document}
